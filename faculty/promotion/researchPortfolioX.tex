\documentclass[11pt,]{article}
\usepackage{lmodern}
\usepackage{amssymb,amsmath}
\usepackage{ifxetex,ifluatex}
\usepackage{fixltx2e} % provides \textsubscript
\ifnum 0\ifxetex 1\fi\ifluatex 1\fi=0 % if pdftex
  \usepackage[T1]{fontenc}
  \usepackage[utf8]{inputenc}
\else % if luatex or xelatex
  \ifxetex
    \usepackage{mathspec}
    \usepackage{xltxtra,xunicode}
  \else
    \usepackage{fontspec}
  \fi
  \defaultfontfeatures{Mapping=tex-text,Scale=MatchLowercase}
  \newcommand{\euro}{€}
    \setmainfont{Georgia}
\fi
% use upquote if available, for straight quotes in verbatim environments
\IfFileExists{upquote.sty}{\usepackage{upquote}}{}
% use microtype if available
\IfFileExists{microtype.sty}{%
\usepackage{microtype}
\UseMicrotypeSet[protrusion]{basicmath} % disable protrusion for tt fonts
}{}
\usepackage[margin=1.0in]{geometry}
\ifxetex
  \usepackage[setpagesize=false, % page size defined by xetex
              unicode=false, % unicode breaks when used with xetex
              xetex]{hyperref}
\else
  \usepackage[unicode=true]{hyperref}
\fi
\hypersetup{breaklinks=true,
            bookmarks=true,
            pdfauthor={},
            pdftitle={},
            colorlinks=true,
            citecolor=blue,
            urlcolor=blue,
            linkcolor=magenta,
            pdfborder={0 0 0}}
\urlstyle{same}  % don't use monospace font for urls
\setlength{\parindent}{0pt}
\setlength{\parskip}{6pt plus 2pt minus 1pt}
\setlength{\emergencystretch}{3em}  % prevent overfull lines
\providecommand{\tightlist}{%
  \setlength{\itemsep}{0pt}\setlength{\parskip}{0pt}}
\setcounter{secnumdepth}{0}

%%% Use protect on footnotes to avoid problems with footnotes in titles
\let\rmarkdownfootnote\footnote%
\def\footnote{\protect\rmarkdownfootnote}

%%% Change title format to be more compact
\usepackage{titling}

% Create subtitle command for use in maketitle
\newcommand{\subtitle}[1]{
  \posttitle{
    \begin{center}\large#1\end{center}
    }
}

\setlength{\droptitle}{-2em}
  \title{}
  \pretitle{\vspace{\droptitle}}
  \posttitle{}
  \author{}
  \preauthor{}\postauthor{}
  \date{}
  \predate{}\postdate{}

\usepackage{booktabs}
\usepackage[font={small},labelfont=bf,labelsep=colon]{caption}
\linespread{1.25}
\usepackage[compact]{titlesec}
\usepackage{enumitem}
\usepackage{tikz}
\def\checkmark{\tikz\fill[scale=0.4](0,.35) -- (.25,0) -- (1,.7) -- (.25,.15) -- cycle;}
\setlist{nolistsep}
\titlespacing{\section}{2pt}{*0}{*0}
\titlespacing{\subsection}{2pt}{*0}{*0}
\titlespacing{\subsubsection}{2pt}{*0}{*0}
\setlength{\parskip}{6pt}

% Redefines (sub)paragraphs to behave more like sections
\ifx\paragraph\undefined\else
\let\oldparagraph\paragraph
\renewcommand{\paragraph}[1]{\oldparagraph{#1}\mbox{}}
\fi
\ifx\subparagraph\undefined\else
\let\oldsubparagraph\subparagraph
\renewcommand{\subparagraph}[1]{\oldsubparagraph{#1}\mbox{}}
\fi

\begin{document}
\maketitle

\pagenumbering{gobble}

\section{Research Portfolio}\label{research-portfolio}

\emph{Nicholas J. Tustison, DSc}

\subsection{I. Research Statement}\label{i.-research-statement}

As research support faculty in the Department of Radiology and Medical
Imaging at UVa, my primary research focus is the development of high
quality and robust software for processing of medical imaging data and
corresponding analysis strategies for these data. To put it simply and
generally, I am a data scientist who operates at the nexus of ``big
data'', statistical science (including machine learning), and software
development for gaining insight into human systems (specifically brain,
lung, and heart).

\subsubsection{A. Major Contributor to The Insight
Toolkit}\label{a.-major-contributor-to-the-insight-toolkit}

Addressing the deficiency of image processing tools for analyzing the
Visible Human Project in 1999, the National Library of Medicine (NLM) of
the National Institutes of Health (NIH) funded the Insight Toolkit (ITK)
initiative bringing together such academic institutions as the
University of Pennsylvania and the University of North Carolina, along
with various industrial partners, such as GE Research. This effort
resulted in the Insight Toolkit---a comprehensive, open-source suite of
codified algorithms for medical image analysis. Development and
expansion continues to the present and is heavily utilized by industry
and academia worldwide and, due to its generalizability, has been
adopted by the French space agency (CNES) for the processing of
remote-sensing imagery.

\textbf{Open source software contributions.} I have provided several key
developments to the Insight Toolkit which is one of the primary venues
for my software contributions and influence in the field. These
contributions are listed in the accompanying CV in Section IX,
Subsection C \emph{Open-Source Software Short Communications}. Starting
from my first contribution in 2005 (\emph{\(N\)-D \(C^k\) B-Spline
Scattered Data Approximation}), I have continued to provide open-source
algorithmic implementations to the Insight Toolkit including my latest
contribution made recently on August 27, 2016 (\emph{Two Luis Miguel
fans walk into a bar in Nagoya ---\textgreater{} (yada, yada, yada)
---\textgreater{} an ITK-implementation of a popular patch-based
denoising filter}). Other important contributions include operations for
image convolution (\emph{Image Kernel Convolution}), faux Colormapping
(\emph{Meeting Andy Warhol Somewhere Over the Rainbow: RGB Colormapping
and ITK}), and fundamental measures for evaluating segmentation results
(\emph{Introducing Dice, Jaccard, and Other Label Overlap Measures To
ITK}). These software classes have been downloaded over 67,413 times
(average = 2,931 downloads).

\textbf{N4 for MRI bias correction.} Of all these contributions, perhaps
my most significant is a method for removing the low frequency
inhomogeneity artifacts common to MR images as an important
preprocessing step for MR image analysis. This algorithm is commonly
referred to in the literature as ``N4'' or ``Nick's nonparametric
nonuniform intensity normalization'' which is described in the following
publication:

\begin{quote}
Tustison NJ, Avants BB, Cook PA, Egan A, Zheng Y, Yushkevich PA, and Gee
JC. N4ITK: Improved N3 Bias Correction, \emph{IEEE Trans Med Imaging},
29(6):1310--1320, June 2010. Cited 367 times; IF = 3.390; Rank 5 out of
100 computer science, interdisciplinary applications, 12 out of 76
biomedical engineering, 18 out of 249 electrical \& electronic
engineering, 3 out of 24 imaging science \& photographic technology, 21
out of 125 radiology, nuclear medicine \& medical imaging.
\end{quote}

It is a significant extension of the popular N3 algorithm\footnote{Sled
  JG, Zijdenbos AP, and Evans AC. A nonparametric method for automatic
  correction of intensity nonuniformity in MRI data. \emph{IEEE Trans
  Med Imaging}, 17(1):87-97, Feb 1998.} (introduced in 1998 with
currently \textasciitilde{}3,000 citations). Prior to the N4 formal
publication, it was provided as open-source software to the ITK
community:

\begin{quote}
Tustison NJ, Gee JC: N4ITK: Nick's N3 ITK Implementation for MRI Bias
Field Correction, \emph{Insight Journal}, 2009,
\url{http://hdl.handle.net/10380/3053}.
\end{quote}

where it has been downloaded over 10,000 times.

\textbf{ITKv4 Image Registration Refactoring.} Image registration (or
the alignment of corresponding features between two images) is a
fundamental component in medical image processing and analysis. In 2011
the NIH-NLM sponsored a large-scale funding effort to ``modernize'' the
Insight Toolkit. One of the three major contracts was to provide modern
image registration techniques requiring a complete refactoring of the
existing image registration framework. This contract was awarded to a
joint team consisting of myself and collaborators from the University of
Pennsylvania (under the direction of Professor James C. Gee):

\begin{quote}
Sponsor: NIH-NLM\\
Title: \emph{Fundamental Refactoring of Deformable Image Registration in
ITK with Distributed Computing and GPU Acceleration}\\
Role: Principle investigator of UVa subcontract\\
Period: 7/1/2011 -- 6/30/2012
\end{quote}

This team provided several major image registration upgrades to the
algorithmic toolkit where I wrote a significant portion of the actual
software code. Not only did we implement current image registration
technologies for inclusion but we also developed new and innovative
techniques which were also included:

\begin{quote}
Tustison NJ and Avants BB. Explicit B-spline regularization in
diffeomorphic image registration. \emph{Front Neuroinform}, 7:39, 2013.
Cited 21 times; IF = 3.261; Rank 8 out of 57 mathematical \&
computational biology, 105 out of 252 neurosciences.
\end{quote}

\begin{quote}
Avants BB, Tustison NJ, Stauffer M, Song G, Wu B, and Gee JC. The
Insight ToolKit Image Registration Framework. \emph{Front Neuroinform},
8:44, 2014. Cited 29 times; IF = 3.261; Rank 8 out of 57 mathematical \&
computational biology, 105 out of 252 neurosciences.
\end{quote}

\textbf{ITK-Lung: A Software Framework for Lung Image Processing and
Analysis.} Consistent with our previous work, Professor James C. Gee and
I recently submitted an NIH R01 grant for the development of ITK-Lung, a
set of open-source software tools for CT, PET, MRI pulmonary image
analysis based on the Insight ToolKit. Specifically, we plan to provide
core algorithms for specific pulmonary image analysis tasks across
multiple modalities, many of which I have included with previous
publications. These basic tasks include intra- and inter-modal pulmonary
image registration, template building for cross-sectional and
longitudinal (i.e., respiratory cycle) analyses, functional and
structural lung image segmentation, perfusion analysis, and computation
of quantitative image indices as potential imaging biomarkers. These
efforts would facilitate other NIH-sponsored projects which interface
specific pulmonary algorithms (e.g., CT nodule detection) with clinical
and research applications. Over the course of this 5-year project, the
following UVa faculty and staff will be engaged:

\begin{itemize}
\tightlist
\item
  Nicholas J. Tustison, DSc, Principal Investigator (50\% / year)
\item
  Kun Qing, PhD, Co-investigator (15\% / year)
\item
  Y. Michael Shim, MD, Co-investigator (2\% / year)
\item
  W. Gerald Teague, MD, Co-investigator (2\% / year)
\end{itemize}

\subsubsection{B. Co-Founder and Developer of the Advanced Normalization
Tools
(ANTs)}\label{b.-co-founder-and-developer-of-the-advanced-normalization-tools-ants}

In the late 2000s my longtime colleague, Dr.~Brian Avants, and I
co-founded the Advanced Normalization Tools (ANTs). ANTs is popularly
considered a state-of-the-art medical image registration and
segmentation toolkit based on ITK. It is used by multiple academic
institutions, research facilities (e.g., the Allen Brain Institute, the
Montreal Neurological Institute, the Laboratory of Neuroimaging at the
University of Southern California), and industry leaders (e.g., IBM
Watson, GE Research). In addition to providing well-performing basic
processing components, we have also engineered advanced pipelines for
obtaining key biomarkers for specific applications. For example,
measuring the thickness of the cortical gray matter of the brain from
MRI has long been used for assessing various neuropathologies and normal
longitudinal changes in the brain. For a long time, the only publicly
available resource for performing this type of measurement was a
software program called ``FreeSurfer'' which is developed and made
available from Mass General Hospital of Harvard University. Recently,
however, I (along with several colleagues) created an ANTs-based
pipeline which outperformed FreeSurfer on a large, publicly available
data set. This work is described in

\begin{quote}
Tustison NJ, Cook PA, Klein A, Song G, Das SR, Duda JT, Kandel BM, van
Strien N, Stone JR, Gee JC, and Avants BB. Large-Scale Evaluation of
ANTs and FreeSurfer Cortical Thickness Measurements. \emph{NeuroImage},
99:166-179, Oct 2014. Cited 46 times; IF = 6.357; Rank 1 out of 14
neuroimaging, 24 out of 252 neurosciences, 3 out of 125 radiology,
nuclear medicine \& medical imaging.
\end{quote}

All resulting quantities and corresponding scripts and analyses were
made publicly available for other people to use. In fact, these
measurements were used recently for investigating other hypotheses:

\begin{quote}
Hasan KM, Mwangi B, Cao B, Keser Z, Tustison NJ, Kochunov P, Frye RE,
Savatic M, and Soares J. Entorhinal cortex thickness across the human
lifespan. J of Neuroimaging, 26(3) :278-82, May 2016. Cited 0 times; IF
= 1.734; Rank 128 out of 192 clinical neurology, 12 out of 14
neuroimaging, and 65 out of 125 radiology, nuclear medicine \& medical
imaging.
\end{quote}

\textbf{Competitions.} Over the years our ANTs-based tools have won
several international competitions:

\begin{itemize}
\item
  finished in the first rank in the Klein 2009 international brain
  mapping competition,\footnote{Klein et al., Evaluation of 14 nonlinear
    deformation algorithms applied to human brain MRI registration.
    \emph{NeuroImage}, 46(3):786-802, Jul 2009.}
\item
  finished first overall in the EMPIRE10 international lung mapping
  competition,\footnote{Murphy et al., Evaluation of registration
    methods on thoracic CT: the EMPIRE10 challenge. \emph{IEEE Trans Med
    Imaging}, 30(11):1901-20, Nov 2011.}
\item
  was the standard registration tool for the MICCAI 2013 segmentation
  competitions,\footnote{\url{http://www.miccai2013.org}}
\item
  finished first in the BRATS 2013 challenge,\footnote{\url{http://martinos.org/qtim/miccai2013/}}
  and
\item
  won the best paper award at the STACOM 2014 challenge.\footnote{\url{http://www.springer.com/us/book/9783319146775}}
\end{itemize}

We have provided these winning protocols to the public as open-source
for continued development.

\textbf{Tutorials and Other ANTs Informational Fora.} I have given
several in-person workshops at the request of various research groups so
that they can better understand the various algorithms and pipelines of
the ANTs toolkit. These include the following:

\begin{itemize}
\item
  ANTs workshop, MD Anderson, Houston, TX, USA. August 2016.
\item
  ANTs Workshop for the Chronic Effects of Neurotrauma Consortium
  (CENC), Baylor College, Houston, TX, USA. October 2015.
\item
  SimpleITK tutorial, MICCAI, Munich, Germany. October 2015.
\item
  ANTs workshop, Laboratory of Neuroimaging, Marina Del Rey, USA. July
  2015.
\item
  CREATE-MIA Summer Workshop, ANTs Workshop, Montreal, Canada. May 2015.
\item
  SPIE Medical Imaging Workshop, Open source tools for medical image
  analysis, San Diego, USA. February 2012.
\end{itemize}

In addition to these workshops, I typically answer 3--4 ANTs inquiries
per day originating from our Sourceforge or Github ANTs repositories.
These inquiries range from instructions for specific programs to
providing analysis guidelines for large-scale studies.

\textbf{ANTsR}

\subsubsection{C. Provide support for UVa
faculty}\label{c.-provide-support-for-uva-faculty}

\begin{itemize}
\item
  Tumor work with Max Wintermark
\item
  Hyperpolarized gas work
\item
  Cardiac work with Mike Salerno
\item
  fMRI processing for Jonathon Kipnis and Mark Beernhaker (also mention
  pending grants)
\end{itemize}

\subsubsection{D. External
collaborations}\label{d.-external-collaborations}

\begin{itemize}
\item
  Work with Jim
\item
  Work with Martha
\item
  Work with Mike
\end{itemize}

\end{document}
