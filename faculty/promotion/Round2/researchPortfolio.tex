% Options for packages loaded elsewhere
\PassOptionsToPackage{unicode}{hyperref}
\PassOptionsToPackage{hyphens}{url}
%
\documentclass[
  11pt,
]{article}
\usepackage{amsmath,amssymb}
\usepackage{lmodern}
\usepackage{iftex}
\ifPDFTeX
  \usepackage[T1]{fontenc}
  \usepackage[utf8]{inputenc}
  \usepackage{textcomp} % provide euro and other symbols
\else % if luatex or xetex
  \usepackage{unicode-math}
  \defaultfontfeatures{Scale=MatchLowercase}
  \defaultfontfeatures[\rmfamily]{Ligatures=TeX,Scale=1}
  \setmainfont[]{Arial}
\fi
% Use upquote if available, for straight quotes in verbatim environments
\IfFileExists{upquote.sty}{\usepackage{upquote}}{}
\IfFileExists{microtype.sty}{% use microtype if available
  \usepackage[]{microtype}
  \UseMicrotypeSet[protrusion]{basicmath} % disable protrusion for tt fonts
}{}
\makeatletter
\@ifundefined{KOMAClassName}{% if non-KOMA class
  \IfFileExists{parskip.sty}{%
    \usepackage{parskip}
  }{% else
    \setlength{\parindent}{0pt}
    \setlength{\parskip}{6pt plus 2pt minus 1pt}}
}{% if KOMA class
  \KOMAoptions{parskip=half}}
\makeatother
\usepackage{xcolor}
\IfFileExists{xurl.sty}{\usepackage{xurl}}{} % add URL line breaks if available
\IfFileExists{bookmark.sty}{\usepackage{bookmark}}{\usepackage{hyperref}}
\hypersetup{
  hidelinks,
  pdfcreator={LaTeX via pandoc}}
\urlstyle{same} % disable monospaced font for URLs
\usepackage[margin=0.75in]{geometry}
\usepackage{graphicx}
\makeatletter
\def\maxwidth{\ifdim\Gin@nat@width>\linewidth\linewidth\else\Gin@nat@width\fi}
\def\maxheight{\ifdim\Gin@nat@height>\textheight\textheight\else\Gin@nat@height\fi}
\makeatother
% Scale images if necessary, so that they will not overflow the page
% margins by default, and it is still possible to overwrite the defaults
% using explicit options in \includegraphics[width, height, ...]{}
\setkeys{Gin}{width=\maxwidth,height=\maxheight,keepaspectratio}
% Set default figure placement to htbp
\makeatletter
\def\fps@figure{htbp}
\makeatother
\setlength{\emergencystretch}{3em} % prevent overfull lines
\providecommand{\tightlist}{%
  \setlength{\itemsep}{0pt}\setlength{\parskip}{0pt}}
\setcounter{secnumdepth}{-\maxdimen} % remove section numbering
\usepackage{booktabs}
\usepackage[font={small},labelfont=bf,labelsep=colon]{caption}
\linespread{1.1}
\usepackage[compact]{titlesec}
\usepackage{enumitem}
\usepackage{tikz}
\def\checkmark{\tikz\fill[scale=0.4](0,.35) -- (.25,0) -- (1,.7) -- (.25,.15) -- cycle;}
\setlist{nolistsep}
\titlespacing{\section}{2pt}{*0}{*0}
\titlespacing{\subsection}{2pt}{*0}{*0}
\titlespacing{\subsubsection}{2pt}{*0}{*0}
\setlength{\parskip}{6pt}
\ifLuaTeX
  \usepackage{selnolig}  % disable illegal ligatures
\fi
\newlength{\cslhangindent}
\setlength{\cslhangindent}{1.5em}
\newlength{\csllabelwidth}
\setlength{\csllabelwidth}{3em}
\newenvironment{CSLReferences}[2] % #1 hanging-ident, #2 entry spacing
 {% don't indent paragraphs
  \setlength{\parindent}{0pt}
  % turn on hanging indent if param 1 is 1
  \ifodd #1 \everypar{\setlength{\hangindent}{\cslhangindent}}\ignorespaces\fi
  % set entry spacing
  \ifnum #2 > 0
  \setlength{\parskip}{#2\baselineskip}
  \fi
 }%
 {}
\usepackage{calc}
\newcommand{\CSLBlock}[1]{#1\hfill\break}
\newcommand{\CSLLeftMargin}[1]{\parbox[t]{\csllabelwidth}{#1}}
\newcommand{\CSLRightInline}[1]{\parbox[t]{\linewidth - \csllabelwidth}{#1}\break}
\newcommand{\CSLIndent}[1]{\hspace{\cslhangindent}#1}

\author{}
\date{\vspace{-2.5em}}

\begin{document}

\pagenumbering{gobble}

\begin{center}
{\Large \bf Research Statement}

{\em Nicholas J. Tustison}
\end{center}

As an interdisciplinary scientist with a comprehensive background in the
field of biomedical imaging analytics, my research endeavors resonate
with both the academic community at the University of Virginia and the
wider scientific community. My primary focus lies in advancing our
understanding of human systems through non-invasive approaches,
employing cutting-edge quantitative imaging techniques. Specifically, my
research centers on the development of robust computational tools
tailored for extracting meaningful insights from complex imaging data. I
also harness statistical methodologies to uncover intricate
relationships between these imaging data and other pertinent biomarkers.
One distinctive aspect of my research is its commitment to open science
principles; in collaboration with fellow researchers, I ensure that all
the tools and resources we develop are openly accessible as open-source
contributions, thereby enhancing the transparency and reproducibility of
our work. Specific contributions are outlined below.

\hypertarget{open-source-software-development-for-medical-image-analysis}{%
\subsection{Open-source software development for medical image
analysis}\label{open-source-software-development-for-medical-image-analysis}}

\hypertarget{the-insight-segmentation-and-registration-toolkit}{%
\subsubsection{The Insight Segmentation and Registration
Toolkit}\label{the-insight-segmentation-and-registration-toolkit}}

The Insight Segmentation and Registration Toolkit (ITK) is an
open-source software library developed primarily by the National Library
of Medicine (NLM) at the National Institutes of Health (NIH).
Researchers and medical professionals use ITK as a quantitative tool in
their work involving medical image analysis, computer-aided diagnosis,
and other applications in the field of medical imaging. I currently play
a significant role in ITK development and maintenance (and have been for
almost two decades where I have been one of the most prolific
contributors to the toolkit). This includes refactoring and ongoing
maintenance of the image registration framework. My image registration
expertise has led to numerous joint research efforts and requests for
collaboration. Over the past year alone, this has resulted in multiple
external collaborations, e.g., determination of the effects of treatment
in brain structures in multiple sclerosis {[}1{]}, the development of
standard image templates to facilitate dementia research in Down
syndrome {[}2{]}, construction of templates from human hand MRI {[}3{]},
and ongoing work with Yongsoo Kim and Fae Kronman (PSU) involving
spatial mapping of mice data. Also see the external funded grant
``Methods for integrative analysis of modern data sources to advance
understanding of Alzheimer's Disease'' (Kristin Linn, Univ. of
Pennsylvania) and external pending grant ``Development of Advanced
Software Tools for Processing Multimodal Medical Images of Healthy and
Diseased Adult Human Hands'' (Jay Hegde, University of Augusta) as well
as previous and current mentoring opportunities---cf. dissertation
committee member for Sebastian Giudice (UVa), Jesse Birchfield (UCLA),
and Daniel Brennan (CUNY).

Additionally, I facilitate the integration of my open-source
contributions to other packages. For example, I developed an algorithm
for removing the low frequency inhomogeneity artifacts common to MR
images---commonly referred to in the literature as ``N4'' or "Nick's
nonparametric nonuniform intensity normalization (the corresponding
journal article continues to be one of the most highly cited
publications in the field with currently close to 5000 total citations
with an increasing citation rate year-over-year and is also ranked in
the top 10, in terms of total number of citations, for articles in IEEE
Transactions on Medical Imaging, one of top-tier journals in the field).
Such major analytic packages (with sponsoring institution) includes:
ANTsX (UPenn/UVa), SimpleITK (National Institutes of Health), Slicer
(Harvard), FreeSurfer (Harvard), Nipype (MIT), fMRIPrep (Stanford
University), Neuroconductor (Johns Hopkins University), MRtrix3 (The
Florey Institute of Neuroscience and Mental Health), QSIPrep (University
of Pennsylvania), and volBrain (Universidad Politécnica de
Valencia/University of Bordeaux). This particular method has also led to
a recent collaborative research projects (e.g, {[}4{]}).

\hypertarget{the-antsx-ecosystem}{%
\subsubsection{The ANTsX Ecosystem}\label{the-antsx-ecosystem}}

I am one of the co-founders and principal developers of the Advanced
Normalization Tools Ecosystem (ANTsX). Since its inception in 2006 with
co-founder Brian Avants, ANTsX has empowered imaging scientists and
engineers, both in academia and industry, to address contemporary
challenges in quantitative biomedical imaging. During my tenure as
associate professor, the ANTsX team leveraged development for successful
application of two explicitly ANTs-based NIH R01 grants:

\begin{quote}
Title: ITK-Lung: A Software Framework for Lung Image Processing and
Analysis\\
Sponsor: NIH/Univ. of Penn\\
Role: Site-PI\\
Dating: 6/1/2017-5/31/2021
\end{quote}

\begin{quote}
Title: Advanced Normalization Tools\\
Sponsor: NIH/Univ. of Penn\\
Role: Site-PI\\
Dating: 9/30/2022-6/30/2027
\end{quote}

In addition to core algorithmic/methodological innovation, I do
significant development for expanding the ANTs framework to other
platforms for the purposes of maximizing utility for a diverse research
community. This includes the development and maintenance of ANTsR (ANTs
in the R Statistical Project) and ANTsPy (ANTs in Python). Additionally,
these ANTs-based platforms have facilitated the current research
direction of developing and integrating deep learning techniques for
biomedical image analysis. Specifically, these have led to the creation
of the ANTsRNet and ANTsPyNet libraries for the R and Python platforms
which have allowed expansion even further into providing
state-of-the-art, high quality measurement tools for scientists.

Collectively, the ANTs software ecosystem is referred to as ANTsX and is
publicly available on GitHub (\url{https://github.com/ANTsX/ANTs}) where
we address bugs, provide documentation, and otherwise engage with
members of the community that use our tools. Related, I participate in
tutorials and workshops to provide a hands-on experience for ANTsX users
(cf.~the Invited Lectures and Symposium of CV). Further discussion of
the ANTsX ecosystem detailing functionality and additional evaluation is
provided in a recent paper {[}5{]} (attached as part of the portfolio).

A sampling of specific workflows that are freely available as part of
the ANTsX toolkit include:

\begin{itemize}
\item
  \emph{Cortical thickness}. First reported for cross-sectional data in
  {[}6{]} (with over 600 citations), this framework was extended to
  longitudinal data in {[}7{]} (attached as part of the portfolio). To
  further enhance accuracy and performance, these pipelines were
  redesigned using deep learning techniques {[}5{]}.
\item
  \emph{DeepFLASH}. Collaborator Michael Yassa (Professor at the
  University of California, Irvine) runs a lab that is broadly
  interested in learning and memory and we collaborate on quantitative
  measurements involving the hippocampus {[}8{]}. We recently developed
  a network, \emph{DeepFLASH}, for parcellation of the medial temporal
  lobe, including the hippocampus and its subfields and
  extra-hippocampal regions. Current efforts include large-scale
  processing and evaluation (UK Biobank, \textasciitilde50,000 subjects
  {[}9{]}) and Parkinson's imaging research (with longtime collaborator,
  Brian Avants).
\item
  \emph{Cerebellum morphology}. Recent work in Parkinson's disease
  research with Brian Avants resulted in a cerebellum morphological
  quantitative pipeline for generating associated image-derived
  phenotypes (IDPs). Given the unique capabilities of ANTsX, these
  cerebelluar IDPs comprise both regional volumes and cortical thickness
  based on the Schmahmann atlas for cerebellar sub-segmentation. This is
  ongoing work also involving the UK Biobank {[}9{]}.
\end{itemize}

\hypertarget{traumatic-brain-injury}{%
\subsection{Traumatic brain injury}\label{traumatic-brain-injury}}

With UVa collaborators James Stone and Brian Avants, we are engaged in
ongoing research investigating chronic neurological changes in
individuals repeatedly exposed to low-intensity blasts, most notably in
military personnel. We have been particularly engaged in the development
of statistical techniques for transforming the large quantities of
information associated with image data to interpretable, low-dimensional
spaces. Recently, our joint efforts have resulted in a powerful
technique called similarity-driven multi-view linear reconstruction
(SiMLR) {[}10{]}. Recent findings have leveraged SiMLR to characterize
the relationship between repetitive low-level blast exposure and
behavioral and imaging differences in humans {[}11{]}. Related funding
includes the following:

\begin{quote}
Title: Interpretable, subject specific-mapping of neurological health in
the performance setting\\
Sponsor: DOD/ONR\\
Role: Co-I\\
Dating: 4/1/2023-3/31/2025
\end{quote}

\begin{quote}
Title: Elucidating the role of increased neuroinflammation and related
structural and functional neurological sequelae after exposure to
repetitive blast\\
Sponsor: CDMRP\\
Role: Co-I\\
Dating: 9/30/2022-9/29/2026
\end{quote}

Our expertise in imaging analytics in the context of traumatic brain
injury is well-known which has led to external collaborations (e.g.,
{[}12{]}) and funding opportunities:

\begin{quote}
Title: Personalized Profiles of Pathology in Pediatric Traumatic Brain
Injury\\
Sponsor: NIH/Univ. of Utah\\
Role: Co-I\\
Dating: 1/1/2022-9/30/2026
\end{quote}

\begin{quote}
Title: Advanced Neuroimaging Analyses for LIMBIC-CENC\\
Sponsor: Veterans Health Admin/Univ. of Utah\\
Role: Co-I\\
Dating: 3/22/2023-3/21/2024
\end{quote}

\hypertarget{functional-lung-imaging-via-hyperpolarized-gas}{%
\subsection{Functional lung imaging via hyperpolarized
gas}\label{functional-lung-imaging-via-hyperpolarized-gas}}

Although most of my application research involves neuroimaging, I have a
longer history with the functional lung imaging research group at UVa.
Current UVa collaborators include Jaime Mata, John Mugler, and Y.
Michael Shim. My role is the development of quantitative techniques for
these innovative image techniques and have been working on this for over
a decade {[}13{]}--{[}15{]} and are applied by in our clinical studies
(recent studies within the past 2 years include {[}16{]}--{[}20{]}).

More recently, I have leveraged deep learning techniques for an updated
approach to quantification {[}21{]}, {[}22{]}. These approaches have
been made publicly available through the ANTsX ecosystem and this work
has led to more wide-scale adoption of these techniques
(\url{http://www.xenonanalysis.com}). Several ongoing and pending grants
include:

\begin{quote}
Title: Sexual dimorphism in susceptibility to emphysematous tissue
injury\\
Sponsor: NIH\\
Role: Co-I\\
Dating: 7/1/2023-6/30/2027
\end{quote}

\begin{quote}
Title: Pilot Study to Determine Health Effects of e-cigarette in Healthy
Young Adults\\
Sponsor: NIH\\
Role: Co-I\\
Dating: 8/20/2020-1/30/2024
\end{quote}

\begin{quote}
Title: Developing Hyperpolarized Gas MRI signatures to detect and manage
acute cellular rejection\\
Sponsor: NIH\\
Role: Co-I\\
Dating: 4/1/2024-3/1/2029
\end{quote}

\begin{quote}
Title: Multi-omic Characterization of COPD in Females\\
Sponsor: NIH\\
Role: Co-I/Penn State Univ.\\
Dating: 4/1/2024-3/31/2028
\end{quote}

\hypertarget{scholarship}{%
\section{Scholarship}\label{scholarship}}

\hypertarget{service}{%
\section{Service}\label{service}}

\begin{itemize}
\tightlist
\item
  BRATS registration competition
\end{itemize}

\hypertarget{references}{%
\section*{References}\label{references}}
\addcontentsline{toc}{section}{References}

\hypertarget{refs}{}
\begin{CSLReferences}{0}{0}
\leavevmode\vadjust pre{\hypertarget{ref-Song:2022aa}{}}%
\CSLLeftMargin{{[}1{]} }
\CSLRightInline{Z. Song \emph{et al.}, {``Deformation-based morphometry
identifies deep brain structures protected by ocrelizumab,''}
\emph{Neuroimage Clin}, vol. 34, p. 102959, 2022, doi:
\href{https://doi.org/10.1016/j.nicl.2022.102959}{10.1016/j.nicl.2022.102959}.}

\leavevmode\vadjust pre{\hypertarget{ref-Queder:2022aa}{}}%
\CSLLeftMargin{{[}2{]} }
\CSLRightInline{N. Queder \emph{et al.}, {``Joint-label fusion brain
atlases for dementia research in down syndrome,''} \emph{Alzheimers
Dement (Amst)}, vol. 14, no. 1, p. e12324, 2022, doi:
\href{https://doi.org/10.1002/dad2.12324}{10.1002/dad2.12324}.}

\leavevmode\vadjust pre{\hypertarget{ref-Hegde:2023aa}{}}%
\CSLLeftMargin{{[}3{]} }
\CSLRightInline{J. Hegdé, N. J. Tustison, W. T. Parker, F. Branch, N.
Yanasak, and L. A. Stumpo, {``An anatomical template for the
normalization of medical images of adult human hands,''}
\emph{Diagnostics (Basel)}, vol. 13, no. 12, Jun. 2023, doi:
\href{https://doi.org/10.3390/diagnostics13122010}{10.3390/diagnostics13122010}.}

\leavevmode\vadjust pre{\hypertarget{ref-Tu:2023aa}{}}%
\CSLLeftMargin{{[}4{]} }
\CSLRightInline{D. Tu \emph{et al.}, {``Automated analysis of low-field
brain MRI in cerebral malaria,''} \emph{Biometrics}, vol. 79, no. 3, pp.
2417--2429, Sep. 2023, doi:
\href{https://doi.org/10.1111/biom.13708}{10.1111/biom.13708}.}

\leavevmode\vadjust pre{\hypertarget{ref-Tustison:2021aa}{}}%
\CSLLeftMargin{{[}5{]} }
\CSLRightInline{N. J. Tustison \emph{et al.}, {``The ANTsX ecosystem for
quantitative biological and medical imaging,''} \emph{Sci Rep}, vol. 11,
no. 1, p. 9068, Apr. 2021, doi:
\href{https://doi.org/10.1038/s41598-021-87564-6}{10.1038/s41598-021-87564-6}.}

\leavevmode\vadjust pre{\hypertarget{ref-Tustison:2014ab}{}}%
\CSLLeftMargin{{[}6{]} }
\CSLRightInline{N. J. Tustison \emph{et al.}, {``Large-scale evaluation
of {ANTs} and {FreeSurfer} cortical thickness measurements,''}
\emph{Neuroimage}, vol. 99, pp. 166--79, Oct. 2014, doi:
\href{https://doi.org/10.1016/j.neuroimage.2014.05.044}{10.1016/j.neuroimage.2014.05.044}.}

\leavevmode\vadjust pre{\hypertarget{ref-Tustison:2019aa}{}}%
\CSLLeftMargin{{[}7{]} }
\CSLRightInline{N. J. Tustison \emph{et al.}, {``Longitudinal mapping of
cortical thickness measurements: An {A}lzheimer's {D}isease
{N}euroimaging {I}nitiative-based evaluation study,''} \emph{J
Alzheimers Dis}, Jul. 2019, doi:
\href{https://doi.org/10.3233/JAD-190283}{10.3233/JAD-190283}.}

\leavevmode\vadjust pre{\hypertarget{ref-Reagh:2018aa}{}}%
\CSLLeftMargin{{[}8{]} }
\CSLRightInline{Z. M. Reagh, J. A. Noche, N. J. Tustison, D. Delisle, E.
A. Murray, and M. A. Yassa, {``Functional imbalance of anterolateral
entorhinal cortex and hippocampal dentate/CA3 underlies age-related
object pattern separation deficits,''} \emph{Neuron}, vol. 97, no. 5,
pp. 1187--1198.e4, Mar. 2018, doi:
\href{https://doi.org/10.1016/j.neuron.2018.01.039}{10.1016/j.neuron.2018.01.039}.}

\leavevmode\vadjust pre{\hypertarget{ref-Tustison:2023aa}{}}%
\CSLLeftMargin{{[}9{]} }
\CSLRightInline{N. J. Tustison \emph{et al.}, {``ANTsX
neuroimaging-derived structural phenotypes of UK biobank,''}
\emph{medRxiv}, 2023, doi:
\href{https://doi.org/10.1101/2023.01.17.23284693}{10.1101/2023.01.17.23284693}.}

\leavevmode\vadjust pre{\hypertarget{ref-Avants:2021aa}{}}%
\CSLLeftMargin{{[}10{]} }
\CSLRightInline{B. B. Avants, N. J. Tustison, and J. R. Stone,
{``Similarity-driven multi-view embeddings from high-dimensional
biomedical data,''} \emph{Nat Comput Sci}, vol. 1, no. 2, pp. 143--152,
Feb. 2021, doi:
\href{https://doi.org/10.1038/s43588-021-00029-8}{10.1038/s43588-021-00029-8}.}

\leavevmode\vadjust pre{\hypertarget{ref-Stone:2020aa}{}}%
\CSLLeftMargin{{[}11{]} }
\CSLRightInline{J. R. Stone \emph{et al.}, {``Functional and structural
neuroimaging correlates of repetitive low-level blast exposure in career
breachers,''} \emph{J Neurotrauma}, vol. 37, no. 23, pp. 2468--2481,
Dec. 2020, doi:
\href{https://doi.org/10.1089/neu.2020.7141}{10.1089/neu.2020.7141}.}

\leavevmode\vadjust pre{\hypertarget{ref-Bigler:2019aa}{}}%
\CSLLeftMargin{{[}12{]} }
\CSLRightInline{E. D. Bigler \emph{et al.}, {``Structural neuroimaging
in mild traumatic brain injury: A chronic effects of neurotrauma
consortium study,''} \emph{Int J Methods Psychiatr Res}, vol. 28, no. 3,
p. e1781, Sep. 2019, doi:
\href{https://doi.org/10.1002/mpr.1781}{10.1002/mpr.1781}.}

\leavevmode\vadjust pre{\hypertarget{ref-Tustison:2010aa}{}}%
\CSLLeftMargin{{[}13{]} }
\CSLRightInline{N. J. Tustison \emph{et al.}, {``Pulmonary kinematics
from tagged hyperpolarized helium-3 MRI,''} \emph{J Magn Reson Imaging},
vol. 31, no. 5, pp. 1236--41, May 2010, doi:
\href{https://doi.org/10.1002/jmri.22137}{10.1002/jmri.22137}.}

\leavevmode\vadjust pre{\hypertarget{ref-Tustison:2010ab}{}}%
\CSLLeftMargin{{[}14{]} }
\CSLRightInline{N. J. Tustison, T. A. Altes, G. Song, E. E. de Lange, J.
P. Mugler 3rd, and J. C. Gee, {``Feature analysis of hyperpolarized
helium-3 pulmonary MRI: A study of asthmatics versus nonasthmatics,''}
\emph{Magn Reson Med}, vol. 63, no. 6, pp. 1448--55, Jun. 2010, doi:
\href{https://doi.org/10.1002/mrm.22390}{10.1002/mrm.22390}.}

\leavevmode\vadjust pre{\hypertarget{ref-Tustison:2011aa}{}}%
\CSLLeftMargin{{[}15{]} }
\CSLRightInline{N. J. Tustison \emph{et al.}, {``Ventilation-based
segmentation of the lungs using hyperpolarized (3)he MRI,''} \emph{J
Magn Reson Imaging}, vol. 34, no. 4, pp. 831--41, Oct. 2011, doi:
\href{https://doi.org/10.1002/jmri.22738}{10.1002/jmri.22738}.}

\leavevmode\vadjust pre{\hypertarget{ref-Qing:2023aa}{}}%
\CSLLeftMargin{{[}16{]} }
\CSLRightInline{K. Qing \emph{et al.}, {``Hyperpolarized xenon-129: A
new tool to assess pulmonary physiology in patients with pulmonary
fibrosis,''} \emph{Biomedicines}, vol. 11, no. 6, May 2023, doi:
\href{https://doi.org/10.3390/biomedicines11061533}{10.3390/biomedicines11061533}.}

\leavevmode\vadjust pre{\hypertarget{ref-Myc:2021aa}{}}%
\CSLLeftMargin{{[}17{]} }
\CSLRightInline{L. Myc \emph{et al.}, {``Characterisation of gas
exchange in COPD with dissolved-phase hyperpolarised xenon-129 MRI,''}
\emph{Thorax}, vol. 76, no. 2, pp. 178--181, Feb. 2021, doi:
\href{https://doi.org/10.1136/thoraxjnl-2020-214924}{10.1136/thoraxjnl-2020-214924}.}

\leavevmode\vadjust pre{\hypertarget{ref-Mata:2021aa}{}}%
\CSLLeftMargin{{[}18{]} }
\CSLRightInline{J. Mata \emph{et al.}, {``Evaluation of regional lung
function in pulmonary fibrosis with xenon-129 MRI,''} \emph{Tomography},
vol. 7, no. 3, pp. 452--465, Sep. 2021, doi:
\href{https://doi.org/10.3390/tomography7030039}{10.3390/tomography7030039}.}

\leavevmode\vadjust pre{\hypertarget{ref-He:2021aa}{}}%
\CSLLeftMargin{{[}19{]} }
\CSLRightInline{M. He \emph{et al.}, {``Characterizing gas exchange
physiology in healthy young electronic-cigarette users with
hyperpolarized 129Xe MRI: A pilot study,''} \emph{Int J Chron Obstruct
Pulmon Dis}, vol. 16, pp. 3183--3187, 2021, doi:
\href{https://doi.org/10.2147/COPD.S324388}{10.2147/COPD.S324388}.}

\leavevmode\vadjust pre{\hypertarget{ref-Garrison:2023aa}{}}%
\CSLLeftMargin{{[}20{]} }
\CSLRightInline{W. J. Garrison \emph{et al.}, {``Lung volume dependence
and repeatability of hyperpolarized 129Xe MRI gas uptake metrics in
healthy volunteers and participants with COPD,''} \emph{Radiol
Cardiothorac Imaging}, vol. 5, no. 3, p. e220096, Jun. 2023, doi:
\href{https://doi.org/10.1148/ryct.220096}{10.1148/ryct.220096}.}

\leavevmode\vadjust pre{\hypertarget{ref-Tustison:2019ac}{}}%
\CSLLeftMargin{{[}21{]} }
\CSLRightInline{N. J. Tustison \emph{et al.}, {``Convolutional neural
networks with template-based data augmentation for functional lung image
quantification,''} \emph{Acad Radiol}, vol. 26, no. 3, pp. 412--423,
Mar. 2019, doi:
\href{https://doi.org/10.1016/j.acra.2018.08.003}{10.1016/j.acra.2018.08.003}.}

\leavevmode\vadjust pre{\hypertarget{ref-Tustison:2021ab}{}}%
\CSLLeftMargin{{[}22{]} }
\CSLRightInline{N. J. Tustison \emph{et al.}, {``Image- versus
histogram-based considerations in semantic segmentation of pulmonary
hyperpolarized gas images,''} \emph{Magn Reson Med}, vol. 86, no. 5, pp.
2822--2836, Nov. 2021, doi:
\href{https://doi.org/10.1002/mrm.28908}{10.1002/mrm.28908}.}

\end{CSLReferences}

\end{document}
