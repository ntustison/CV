% Options for packages loaded elsewhere
\PassOptionsToPackage{unicode}{hyperref}
\PassOptionsToPackage{hyphens}{url}
%
\documentclass[
  11pt,
]{article}
\usepackage{amsmath,amssymb}
\usepackage{lmodern}
\usepackage{iftex}
\ifPDFTeX
  \usepackage[T1]{fontenc}
  \usepackage[utf8]{inputenc}
  \usepackage{textcomp} % provide euro and other symbols
\else % if luatex or xetex
  \usepackage{unicode-math}
  \defaultfontfeatures{Scale=MatchLowercase}
  \defaultfontfeatures[\rmfamily]{Ligatures=TeX,Scale=1}
  \setmainfont[]{Arial}
\fi
% Use upquote if available, for straight quotes in verbatim environments
\IfFileExists{upquote.sty}{\usepackage{upquote}}{}
\IfFileExists{microtype.sty}{% use microtype if available
  \usepackage[]{microtype}
  \UseMicrotypeSet[protrusion]{basicmath} % disable protrusion for tt fonts
}{}
\makeatletter
\@ifundefined{KOMAClassName}{% if non-KOMA class
  \IfFileExists{parskip.sty}{%
    \usepackage{parskip}
  }{% else
    \setlength{\parindent}{0pt}
    \setlength{\parskip}{6pt plus 2pt minus 1pt}}
}{% if KOMA class
  \KOMAoptions{parskip=half}}
\makeatother
\usepackage{xcolor}
\IfFileExists{xurl.sty}{\usepackage{xurl}}{} % add URL line breaks if available
\IfFileExists{bookmark.sty}{\usepackage{bookmark}}{\usepackage{hyperref}}
\hypersetup{
  hidelinks,
  pdfcreator={LaTeX via pandoc}}
\urlstyle{same} % disable monospaced font for URLs
\usepackage[margin=0.75in]{geometry}
\usepackage{graphicx}
\makeatletter
\def\maxwidth{\ifdim\Gin@nat@width>\linewidth\linewidth\else\Gin@nat@width\fi}
\def\maxheight{\ifdim\Gin@nat@height>\textheight\textheight\else\Gin@nat@height\fi}
\makeatother
% Scale images if necessary, so that they will not overflow the page
% margins by default, and it is still possible to overwrite the defaults
% using explicit options in \includegraphics[width, height, ...]{}
\setkeys{Gin}{width=\maxwidth,height=\maxheight,keepaspectratio}
% Set default figure placement to htbp
\makeatletter
\def\fps@figure{htbp}
\makeatother
\setlength{\emergencystretch}{3em} % prevent overfull lines
\providecommand{\tightlist}{%
  \setlength{\itemsep}{0pt}\setlength{\parskip}{0pt}}
\setcounter{secnumdepth}{-\maxdimen} % remove section numbering
\usepackage{booktabs}
\usepackage[font={small},labelfont=bf,labelsep=colon]{caption}
\linespread{1.05}
\usepackage[compact]{titlesec}
\usepackage{enumitem}
\usepackage{tikz}
\def\checkmark{\tikz\fill[scale=0.4](0,.35) -- (.25,0) -- (1,.7) -- (.25,.15) -- cycle;}
\setlist{nolistsep}
\titlespacing{\section}{2pt}{*0}{*0}
\titlespacing{\subsection}{2pt}{*0}{*0}
\titlespacing{\subsubsection}{2pt}{*0}{*0}
\setlength{\parskip}{6pt}
\ifLuaTeX
  \usepackage{selnolig}  % disable illegal ligatures
\fi

\author{}
\date{\vspace{-2.5em}}

\begin{document}

\pagenumbering{gobble}

\textbf{Nicholas J. Tustison, DSc}

I am a co-founder and developer of the widely recognized open-source
Advanced Normalization Tools Ecosystem (ANTsX) which allows me to
simultaneously engage with learners and contribute to ongoing research.
This has led to significant opportunities for mentoring and teaching
students, post-graduate trainees, and other researchers by providing
numerous tutorials and workshops at various conferences and at different
academic and scientific institutions. I have also organized this
tutorial material for online public access. This is in addition to my
daily participation in the ANTsX online user forums which allows me to
fulfill my passion for educating the learner.

I adhere to and believe in the ASPIRE values. Indeed, my approach to
research and collaboration is guided by a deep commitment to the
development of high-quality, open-source computational strategies for
biological and medical imaging. I believe that such commitments create a
supportive and inclusive environment where knowledge is shared, diverse
perspectives are valued, and the quality of work is held to the highest
standards. Since being promoted to associate professor, I have been
\textgreater90\% funded and it is projected that I will be 100\% funded
for FY 2024. Funding has come from several collaborative sources,
demonstrating my contribution to and benefit from my academic approach
where diverse perspectives are valued, team science and collaborative
interactions are the norm, and the quality of work is held to the
highest standards.

\textbf{Personal Statement}

I am one of the co-founders and main developers of the Advanced
Normalization Tools Ecosystem (ANTsX)---a state-of-the-art, open-source
library of software tools for image registration, segmentation, and
other quantitative medical imaging functionality. Over the course of its
development, ANTsX has enabled hundreds of academic and industrial
scientists to meet modern quantitative imaging needs with particular
focus on issues in biomedical imaging. A broad range of ANTsX-based
applications and published research sample the study of organisms from
small animals to humans as well as target organ systems such as
respiratory, cardiovascular, and nervous. This toolkit is used widely by
multiple universities (e.g., Stanford University, Harvard University,
University of California, Los Angeles), businesses (e.g., General
Electric Research and Konica Minolta), and research institutions (e.g.,
the Child Mind Institute and the Allen Institute for Brain Science)
around the world. ANTsX has been integrated into multiple, highly vetted
workflows such as fMRIprep (Stanford University) and the Spinal Cord
Toolbox (École Polytechnique de Montréal). Popular ANTsX pipelines, such
as cortical thickness estimation, have been integrated into Docker
containers and packaged as Brain Imaging Data Structures (BIDS) and
FlyWheel applications (i.e., ``gears''). It has also been independently
ported for various platforms including Neurodebian (Debian OS),
Neuroconductor (the R statistical project), and Nipype (Python).
Recently, the ANTsX team has leveraged development in the successful
application of two explicitly ANTs-based NIH R01 grants.

ANTsX software tools that I have implemented with my co-developers are
described by some of the most highly cited publications in the field.
Historically, ANTsX is rooted in advanced image registration techniques
that date back to the earliest seminal work of pioneers in the field.
One of our early implementation papers describing the ANTsX open-source
image registration tool has close to 4000 citations alone:

\begin{itemize}
\tightlist
\item
  Avants BB, \textbf{Tustison NJ}, et al.~A Reproducible Evaluation of
  ANTs Similarity Metric Performance in Brain Image Registration.
  Neuroimage, 54(3):2033--2044, February 2011.
\end{itemize}

Given the reputation of ANTsX performance standards, I am frequently
sought out for collaborations in a consulting or mentoring role, as
attested by my list of publications and shared grants. I have also
provided evaluative comparison data for international competitions due
ANTsX history of superb performance. As recently as last year, I was
asked to provide generated comparative image registration results for a
long running image analysis challenge involving brain tumor data (MICCAI
BraTS Registration Challenge).

The most widely cited ANTsX paper is one that I wrote earlier in my
career detailing the now extremely well-known ``N4'' bias correction
algorithm which has close to 5000 citations with an increasing citation
rate year-over-year.

\begin{itemize}
\tightlist
\item
  \textbf{Tustison NJ}, et al.~N4ITK: Improved N3 Bias Correction. IEEE
  Trans Med Imaging, 29(6):1310--1320, June 2010.
\end{itemize}

N4, an acronym for ``Nick's nonparametric nonuniform intensity
normalization,'' is an algorithm used to ``clean'' an image prior to
computational processing and is considered by many to be a mandatory
step for achieving good results. It has been adopted within many
standard image processing protocols, including those of ``competing''
packages, such as the widely used FreeSurfer software suite of Harvard
University.

More recently, ANTsX has been extended to complementary frameworks
resulting in the Python- and R-based ANTsPy and ANTsR toolkits,
respectively. These packages interface with extremely popular,
high-level, open-source programming platforms which have significantly
increased the user base of ANTs. The rapidly rising popularity of deep
learning motivated further recent enhancement of ANTs and its
extensions, specifically ANTsRNet and ANTsPyNet, dynamic
Keras/TensorFlow-based library of popular deep learning architectures
and applications specifically geared towards medical imaging detailed by
a more recent, but highly cited publication:

\begin{itemize}
\tightlist
\item
  \textbf{Nicholas J. Tustison}, et al.~The ANTsX ecosystem for
  quantitative biological and medical imaging. Sci Rep.~11(1):9068, Apr
  2021.
\end{itemize}

Two sets of collaborations illustrate the utility and wide-applicability
of my work:

\emph{Traumatic Brain Injury.} With UVa collaborators James Stone and
Brian Avants, we are engaged in ongoing research investigating chronic
neurological changes in individuals repeatedly exposed to low-intensity
blasts. We have made major contributions to the community in the form of
statistical methods and analysis of neuroimaging sequelae:

\begin{itemize}
\tightlist
\item
  James Stone, Brian Avants, \textbf{Nicholas Tustison}, et
  al.~Functional and structural neuroimaging correlates of repetitive
  low-level blast exposure in career breachers. J Neurotrauma,
  37(23):2468-2481, Dec 2020.
\end{itemize}

\emph{Functional lung imaging via hyperpolarized gas.} UVa is
internationally recognized for its innovation in functional lung imaging
using hyperpolarized gas. With technical and clinical collaborators such
as John Mugler, Mike Shim, and Jaime Mata, I contribute to the
overarching research goals in terms of algorithmic innovation for image
quantitation and analysis. In fact, the impact of my work has led to
field-wide adoption of my open-source contributions
(www.xenonanalysis.com).

\begin{itemize}
\tightlist
\item
  \textbf{Nicholas J. Tustison}, et al.~Image- versus histogram-based
  considerations in semantic segmentation of pulmonary hyperpolarized
  gas images. Magn Reson Med, 86(5):2822-2836, Nov 2021.
\end{itemize}

In terms of my academic service:

\begin{itemize}
\tightlist
\item
  Since 2018, I have served as the Secretary of the Insight Software
  Consortium. This body provides direction and oversight of the Insight
  Toolkit.
\item
  I am one of the top contributors to the Insight Toolkit of the
  National Library of Medicine under the National Institute of Health
  where I contributed to software for image registration, segmentation,
  visualization, and other processing components. I also provide user
  guidance on the discussion forums.
\item
  Given my expertise and academic reputation, I frequently review for
  various manuscripts and abstracts (average \textgreater{} 1 per month)
  for several highly reputable journals and conferences.
\end{itemize}

\end{document}
